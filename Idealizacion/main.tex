\documentclass{article}
\usepackage[utf8]{inputenc}
\usepackage[spanish]{babel}
\usepackage{listings}
\usepackage{graphicx}
\graphicspath{ {images/} }
\usepackage{cite}

\begin{document}

\begin{titlepage}
    \begin{center}
        \vspace*{1cm}
            
        \Huge
        \textbf{IDEA DEL JUEGO}
            
        \vspace{0.5cm}
        \LARGE
        
            
        \vspace{1.5cm}
            
        \textbf{Jackh Emmanuel Narvaez Guerra}
        
        \textbf{Julián Benigno Méndez Higuera} 
         
        \vfill
            
        \vspace{0.8cm}
            
        \Large
        Despartamento de Ingeniería Electrónica y Telecomunicaciones\\
        Universidad de Antioquia\\
        Medellín\\
        Marzo de 2021
            
    \end{center}
\end{titlepage}

\tableofcontents
\newpage
\section{Introducción}\label{intro}
Vamos a realizar una breve exposición de las ideas que tenemos para el juego del proyecto final. Teniendo en cuenta el corto plazo que tenemos para elaborar un videojuego y también que muchas de estas ideas pueden cambiar a lo largo del proyecto.

\section{Idea del juego} \label{contenido}
- El juego lo desarrollaremos en modelo 2d.

-Tendrá un diseño retro como podría ser Super Mario, Mark of the Ninja, Metal Slug, Mega Man X, etc.

- La temática del juego “La huida de un recluso de la cárcel” en donde tendrá que atravesar por diferentes niveles de la cárcel, en los niveles habrá diferentes desafíos que se deberán superar para avanzar en el juego.

- Intentaremos implementar diferentes herramientas que puede usar el prisionero para escapar de la cárcel, de tal manera que el juego no solo consista en pasar un nivel linealmente si no que se vuelva más dinámico, pero a la vez complicado y así un ni el no tendrá nada que ver con el anterior.

- Otra idea que se nos ocurrió es que podemos llevar a cabo un modo multijugador, de forma que podríamos complicar el juego aun más, ya que no solo se dependerá de un jugador si no también de la astucia que pueda tener el compañero.

\end{document}
